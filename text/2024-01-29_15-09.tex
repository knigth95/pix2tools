1. Introduction
1.1 Background
A variety of sandcastles can be found on the seashore, range from simple mounds
of sand to complex castle replicas. In all these, one typically forms an initial foundation
consisting of a single, nondescript mound of wetted sand, and then proceeds to cut and
shape this base into a recognizable 3-dimensional geometric shape, thereby building
more castle-defining features.
Over time, there is no doubt that rain and waves will gradually erode the sandcastle,
However, the degree of erosion of different types of sandcastles is different. Even if the
building size and the distance from the water on the same beach are roughly same
Therefore, in order to achieve highest robustness and longest lasting time, we
wonder if there exits an optimal 3D geometric shape to use as the foundation of a
sandcastle.
1.2 Our work
To further present our solutions, we arrange our paper as follows:
In task 1, we use six common geometric shapes for research. By establishing
tide immersion model and wave erosion model, we calculate that the cuboid is
the optimal geometric model. Further research, through the study of cuboids
with different aspect ratios, we find that the smaller the width of the cuboid
facing the water, the longer the model exists.
In task 2, different sand-to-water mixing ratios will affect the strength of
sandcastles. By establishing the function relationship between different sand-
to-water mixing ratios and internal friction angles, we find 15% is the best
In task 3, we divide the impact of rainwater on the sandcastle into two parts:
scour and infiltration. Through ANSYS simulation. we get the effects of
rainfall on the six geometries, which are very similar to the theory
In task 4, we consider two aspects of reducing seawater infiltration and
reducing wave erosion to extend the lasting time of the sandcastle
2. Assumptions
Assume that the bottom of the model is at a horizontal plane
Suppose the waves are moving without amplitude attenuation
Disregard the impact of sandcastle's own gravity