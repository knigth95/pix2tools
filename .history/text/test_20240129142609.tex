1.太阳高度角α。[3]
$$
\sin\alpha_{s}=\cos\delta\cos\varphi\cos\omega+\sin\delta\sin\varphi
$$
太阳方位角y。[4]
$$
\cos\gamma_{s}={\frac{\sin\delta-\sin\alpha_{s}\sin\varphi}{\cos\alpha_{s}\cos\varphi}}
$$
其中p为当地纬度,北纬为正;α为太阳时角
$$
\omega={\frac{\pi}{12}}(S T-12),
$$
其中ST 为当地时间,6为太阳赤纬角[5]
$$
\sin\delta=\sin\frac{2\pi D}{365}\sin\left(\frac{2\pi}{360}23.45\right),
$$
其中D为以春分作为第0天起算的天数,例如,若春分是3月21日,则4月1日对应
 $D=11.$ 
2.法向直接辐射辐照度DNI(单位:kW/m-)是指地球上垂直于太阳光线的平面单位面积
上、单位时间内接收到的太阳辐射能量,可按以下公式近似计算[6
$$
\begin{array}{l}{\mathrm{DNI}=G_{0}\left[a+b\exp\left(-\frac{c}{\sin\alpha_{e}}\right)\right],}\\ {a=0.4.37-0.00821(6-H)^{2},}\\ {=0.505+0.0088667-H^{2},}\\ {c=0.2711+0.01858(2.57-H)^{2},}\end{array}
$$
其中
 $G_{\mathrm{0}}$ 
为太阳常数,其值取为1.366 kW/m?,H 为海拔高度(单位:km)。
3.定日镜场的输出热功率
 $E_{\mathrm{frield}}\ {\mathcal{N}}_{J}$ 
$$
E_{\mathrm{field}}=\mathrm{DNI}\sim\sum_{i}^{N}A_{i}\eta_{i},
$$
其中 DNI 为法向直接辐射辐照度;N为定日镜总数(单位:面);A:为第i面定日镜采光
面积(单位:m-);
 $\eta_{i}$ 

为第i面镜子的光学效率。
4.定日镜的光学效率,为
$$
\eta=\eta_{s\mathrm{b}}\eta_{\mathrm{cos}}\eta_{\mathrm{at}}\eta_{\mathrm{trunc}}\eta_{\mathrm{ref}},
$$
其中
法向辐照度表示地球上垂直于太阳光线的平面单位面积上、单位时间内接收到的太阳辐射能量,计
算公式(14)如下:
$$
\begin{array}{c}{{D N I=G_{0}\left[a+b\exp\left(\frac{-c}{\sin\alpha_{s}}\right)\right],}}\\ {{a=0.4237-0.00821(6-H)^{2}}}\\ {{b=0.505+0.00566.5-H)^{2}}}\\ {{c=0.2711+0.01858(2.5-H)^{2}}}\end{array}
$$
(14)